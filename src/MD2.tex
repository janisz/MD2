\documentclass[12pt,a4paper]{article}

\usepackage[utf8]{inputenc}
\usepackage[T1]{fontenc}
\usepackage[a4paper]{geometry}
\usepackage{fullpage}
\usepackage{pdflscape}
\usepackage{float}

\usepackage{amsthm}
\usepackage{amsfonts}
\usepackage{amsmath} 		%pakiet potrzebny do \eqref
\usepackage{amssymb} 		%pakiet potrzebny do \blacksquare i \mathbb{}

\usepackage[pdfauthor={Janisz},%
	pdftitle={MD2},%
	pagebackref=true,%
	pdftex]{hyperref}
\hypersetup{colorlinks=true}%kolorowe linki

\usepackage{graphicx}		%do obrazków
\usepackage[polish]{babel}	%babel musi być po ams inaczej generuje błedy https://groups.google.com/d/msg/pl.comp.dtp.tex.gust/AfdVxQVu_DI/Fego6Pxz3HMJ

%-----------------------DEFINICJE------------------------------------------
\newtheorem{lemat}{Lemat}
\newtheorem{tw}{Twierdzenie}
\newtheorem{przyklad}{Przykład}
\theoremstyle{definition}
\newtheorem{df}{Definicja}
\newtheorem{wniosek}{Wniosek}
\newtheorem{uwaga}{Uwaga}
\newtheorem{problem}{Problem}
\newtheorem{algo}{Algorytm}
\newcommand{\egz}{\marginpar{\color{red} EGZAMIN}} %gneruje ostrzeżenia?
%-------------------------------------------------------------------------

\title{ Matematyka Dyskretna 2}
\author{Tomasz Janiszewski}
\date{\today}

\begin{document}

\maketitle

\tableofcontents

\DeclareGraphicsExtensions{.pdf,.png,.jpg}
\begin{center}
\leavevmode
\includegraphics[width=1 in]{../img/by-sa.png}
\end{center}
\label{fig:cc}
%insert a link to the licence and its description below
\scriptsize{Ten utwór jest dostępny na licencji  \href{http://creativecommons.org/licenses/by-sa/3.0/pl/}{Creative Commons Uznanie autorstwa-Na tych samych warunkach 3.0 Polska.}}

\pagebreak
%---------------%
%1. Wyklad      %
%	4.10.2011   %
%---------------%
\section{Ścieżka i cykl \href{http://pl.wikipedia.org/wiki/Leonhard_Euler}{Eulera}}
%TODO Dodać opis problemu
\begin{df}
Ścieżką Eulera w grafie $G$ nazywamy ścieżkę $v_0e_1v_1e_2 \dots e_nv_n$ taką że każda krawędź występuje na niej tylko raz
\end{df}
\begin{df}[Obwód Eulera]~\\
Jeżeli w danej $v_0 = v_n$ to ścieżkę nazywamy ścieżką zamkniętą lub obwodem Eulera
\end{df}
\begin{df}[Cykl Eulera]~\\
Jeżeli wierzchołki  w obwodem Eulera się nie powtarzają to taki obwód nazywamy cyklem Eulera.
\end{df}

\begin{lemat}\label{lemat:1}
Jeżeli $G$ jest grafem takim że $\delta (G) \geqslant 2$ to $G$ posiada cykl
\begin{proof}
Rozważmy najdłuższą drogę (wierzchołki się nie powtarzają) $v_0e_1v_1 \dots e_kv_k$ oraz niech $\deg v_0 \geqslant \delta(G) \geqslant 2$. Niech sąsiad $v_0 \neq v_1$ nazywa się $a$. $a \not\in \{v_2, \dots, v_k\}$ bo gdyby należał do ścieżki to $av_0v_1 \dots v_k$ byłaby dłuższa niż $v_0 \dots v_k$ wbrew założeniu. Czyli $a\in \{v_2, \dots, v_k\}$ czyli $a = v_i$.\\ Mamy cykl! 
%TODO Dodać ilustrację
\end{proof}
\end{lemat}

\begin{tw}[Eulera]\label{tw:Eulera}~\\
Graf spójny ma obwód Eulera $\Leftrightarrow$ wszystkie wierzchołki mają parzyste stopnie
\begin{proof}
$\Rightarrow$\\
Przemierzając obwód Eulera wchodzimy i wychodzimy do każdego wierzchołka tyle samo razy za każdym razem inną krawędzią. Ponieważ każda krawędź leży w obwodzie Eulera to każdy wierzchołek jest parzystego stopnia
\end{proof}
\begin{proof}
$\Leftarrow$ \emph{nie wprost}\\
Przypuśćmy że twierdzenie nie jest prawdziwe. Ze wszystkich kontrprzykładów weźmy ten który ma najmniej krawędzi, nazwijmy go $G$. To znaczy $G$ jest spójny, wszystkie wierzchołki ma parzystego stopnia a nie ma obwodu Eulera. To znaczy że $|G| > 1$\footnote{graf o jednym wierzchołku ma obwód Eulera} oraz żaden wierzchołek nie jest stopnia 0 lub 1, każdy jest co stopnia co najmniej 2. Czyli z lematu \ref{lemat:1} $G$ ma cykl, a cykl $\subset$ obwód.
Weźmy najdłuższy obwód w $G$ i nazwijmy $C$. $E(C) \neq E(G)$ bo w przeciwnym przypadku $C$ byłby obwodem Eulera w $G$.
\\Rozważmy $G-E(C)$ zawiera jakieś krawędzie. Zastanówmy się nad parzystością stopni $G-E(C)$. W $G-E(C)$ każdy wierzchołek jest parzystego stopnia. czyli każda składowa o nieparzystym zbiorze krawędzi\footnote{$G-E(C)$ nie musi być grafem spójnym} $G-E(C)$ nie jest kontrprzykładem ($G$ był najmniejszy) zatem ma obwód Eulera $C'$.\\
Istnieje wierzchołek $x\in V(C) \cap V(C')$ bo w przeciwnym przypadku $G$ nie byłby spójny.\\
Obwód $xCxC'x$ jest dłuższy niż $C$, a $G$ maił być najdłuższy (o większej liczbie krawędzi). Sprzeczność z definicją $C$.
\end{proof}
%TODO Sprawdzić ten dowód
\end{tw}

\subsection{Algorytm znajdowania obwodu Eulera w grafie}
\begin{lemat}\label{lemat:2}
Jeżeli graf $G$ jest spójny i ma wszystkie wierzchołki parzystego stopnia, to nie ma mostu
\begin{proof}
Przypuśćmy przeciwnie.\\
Niech $e = xy$ będzie mostem w $G$. $G-e$ ma dokładnie dwie składowe $G_x, G_y$. W $G_x$ wszystkie wierzchołki oprócz $X$ są parzystego stopnia. Sprzeczność z zasadą, że suma stopni w grafie jest parzysta (suma stopni jest dwa razy większa od liczby krawędzi)
\end{proof}
\end{lemat}

\begin{algo}[Algorytm Fleury'ego]
\begin{enumerate}
\item Wybieram wierzchołek startowy $v_0, ~ w = v_0$
\item Załóżmy, że skonstruowaliśmy już ścieżkę $w=v_0e_1v_1 \dots e_iv_i$ wybieram krawędź $e_{i+1}$ taką że
	\begin{enumerate}
		\item $e_{i+1} \in E(G) \smallsetminus \{e_1, \dots, e_i\}$
		\item $e_{i+1}$ jest incydentna z $v_i$
		\item o ile to możliwe $e_{i+1}$ nie jest mostem w $G_i = G - \{e_1, \dots, e_i\}$\label{point:1}
	\end{enumerate}
\item STOP kiedy krok 2 nie może się wykonać
%TODO Dodać przykład
\end{enumerate}
\end{algo}

\begin{tw}
Jeżeli $G$ ma obwód Eulera to algorytm Fleury'ego znajdzie taki obwód
\begin{proof}
$w_n = v_0e_1v_1 \dots e_nv_n$ ścieżka skonstruowana przez algorytm Fleury'ego. $w_n$ nie zawiera żadnej krawędzi więcej niż raz. W momencie zatrzymania się algorytmu stopnień $v_n$ w grafie $G-\{e_1, \dots, e_n\}$ jest równy $0$.\\
Jeżeli $v_n = v_0$ to ma nieparzysty stopień, a ma mieć $0$. Czyli $v_n = v_0$\footnote{gdyby tak nie było oznaczałoby to  że wyszliśmy o jeden raz więcej niż weszliśmy} Czyli $\deg_{G-\{e_1, \dots, e_n\}}v_n$ jest nieparzysty. Sprzeczność bo zero jest parzyste.\\
Wystarczy pokazać, że $E(w_n)=E(G)$.\\
Przypuśćmy, że nie jest. $G_n = G-E(w_n)$ ma niepusty zbiór krawędzi.
\\ Niech $S$ - zbiór wierzchołków dodatniego stopnia w $G_n$.
\\ Niech $\overline{S}$ - zbiór wszystkich wierzchołków zerowego stopnia. $\overline{S}=V(G)-S$
\\ Niech $m$ będzie największym indeksem takim że $V_m \in S$ a $V_{m+1} = e_{m+1}$
%TODO Dodać obrazek
\\ Czyli krawędź $e_{m+1}$ jest mostem w $G_m = G - \{e_1, \dots, e_m\}$ bo jest jedyną krawędzią między $S$ i $\overline{S}$. Z punkty \ref{point:1} wynika że wszystkie krawędzie w $G_m$ o końcu $v_m$ są mostami.
\\ Rozważmy $G_m[S]$, w którym wszystkie wierzchołki są parzystego stopnia a $v_m$ jest końcem mostu. Sprzeczność z lematem \ref{lemat:2}.
%TODO Porównać z http://pl.wikipedia.org/wiki/Algorytm_Fleury%27ego#Dow.C3.B3d_poprawno.C5.9Bci_algorytmu
\end{proof}
\end{tw}

%---------------%
%2. Wyklad      %
%	11.10.2011  %
%---------------%
\subsection{Wniosek z twierdzenia Eulera}
\begin{tw}
Spójny multigraf $G$ ma ścieżkę Eulera\footnote{nie musi się kończyć tam gdzie się zaczęła} $\Leftrightarrow~ G$ ma co najwyżej 2 wierzchołki nieparzystego stopnia.
\begin{proof}
$\Rightarrow$\\
Idąc ścieżką Eulera wchodzimy do każdego wierzchołka być może z wyjątkiem pierwszego i ostatniego tyle samo razy co wychodzimy, zatem każdy wierzchołek (może z wyjątkiem pierwszego i ostatniego) ma parzysty stopień.
\end{proof} 
\begin{proof}
$\Leftarrow$
	\begin{enumerate}
		\item Jeśli jest 0 wierzchołków nieparzystego stopnia to z twierdzenia \hyperref[tw:Eulera]{Eulera} istnieje obwód Eulera, który jest ścieżką.
		\item Nie może być dokładnie jeden wierzchołek nieparzystego stopnia, bo suma stopni musi być parzysta.
		\item Jeżeli są 2 takie wierzchołki połączmy je krawędzią. Otrzymujemy multigraf o wszystkich wierzchołkach parzystego stopnia. Z twierdzenia \hyperref[tw:Eulera]{Eulera} ma on obwód Eulera. Usuwając uprzednio dodaną krawędź otrzymujemy ścieżkę  Eulera.
	\end{enumerate}
\end{proof}
\end{tw}

\subsection{Problem chińskiego listonosza}
Listonosz musi dostarczyć listy do adresatów, czyli przejść przez każdą ulicę w mieście. Jak ma przejść aby pokonać najkrótszą trasę?\\
%TODO Dodać rysunek
\subsubsection{Dane} Spójny multigraf $G$, funkcja wag krawędzi (długość ulic) $w : E(G) \to [0,\infty)$\\
\subsubsection{Problem} Znaleźć trasę, czyli ciąg wierzchołków taki że:
\begin{enumerate}
	\item każda krawędź występuje w ciągu
	\item $\forall i \quad e_i = v_{i-1}v_i$
	\item $\sum\limits_{i=1}^{p} w(e_i)$ była minimalna 
\end{enumerate}
\subsubsection{Rozwiązanie}
\begin{enumerate}
	\item Połączy krawędziami wierzchołki nieparzystych stopni po dwa tak aby suma wag u nowych krawędzi była jak najmniejsza\footnote{algorytm jest skomplikowany}
	\item Znajdźmy obwód Eulera w nowym multigrafie
\end{enumerate}
%TODO Poprawić formatowanie problemów

\section{Cykle i drogi \href{http://pl.wikipedia.org/wiki/William_Rowan_Hamilton}{Hamiltona}}

\begin{df}[Droga Hamiltona]~\\
Drogę która zawiera wszystkie wierzchołki grafu $G$ nazywamy drogą Hamiltona
\end{df}

\begin{df}[Cykl Hamiltona]~\\
Cykl który zawiera wszystkie wierzchołki grafu $G$ nazywamy cyklem Hamiltona
\end{df}


%TODO Dodać przykłady

\begin{tw}
Jeżeli $G$ ma cykl Hamiltona to dla każdego $S\subset V(G) \quad \omega(G-S) \leqslant |S|$
\begin{proof}
Niech $C$ - cykl Hamiltona w $G$. Zauważmy, że $\omega(C-S) \leqslant |S|$ \\
$\omega(G-S) \leqslant \omega(C-S)$
\end{proof}
%TODO Dodać przykłady (szczególnie graf Petersena)
\end{tw}

\begin{tw}[\href{http://pl.wikipedia.org/wiki/Gabriel_Andrew_Dirac}{Diraca} (1952)]\egz
Jeżeli $G$ jest grafem takim że $G\geqslant 3, ~~ \delta(G)\geqslant \frac{|G|}{2}$ to $G$ ma cykl Hamiltona
\begin{proof}\emph{nie wprost}\\
Niech $G$ będzie kontrprzykładem do twierdzenia o maksymalnym zbiorze krawędzi, czyli grafem takim że ma co najmniej 3 wierzchołki $|G|\geqslant 3, ~~ \delta(G) \geqslant \frac{|G|}{2}$ nie ma cyklu Hamiltona i dodanie dowolnej krawędzi do $G$ powoduje powstanie cyklu Hamiltona\footnote{dodajemy tak długo jak się da}. Czyli graf $G$ nie jest grafem pełnym bo graf pełny zawiera cykl Hamiltona.\\
W $G$ istnieją wierzchołki $u$ i $v$ takie że $uv\not\in E(G)$\\
Z maksymalności $G$ wynika żę $u$ i $v$ są połączone drogą Hamiltona \footnote{bo $G+uv$ ma cykl Hamiltona to $G=(G+uv)-uv$ ma drogę Hamiltona} $u = v_0,v_1,\dots, v_n = v$ - droga\\
Zdefiniujmy dwa zbiory:
	\begin{enumerate}
		\item $T := \{ v_i: \quad v_iv\in E(G)\}$
		\item $S := \{ v_i: \quad uv_{i+1} \in E(G)\}$
	\end{enumerate}
%TODO Dodać rysunek
\begin{math}
|S| = \deg u \geqslant \frac{|G|}{2}\\
|T| = \deg v \geqslant \frac{|G|}{2}\\
v \not\in S \cup T \Rightarrow |S \cup T| \leqslant |G| -1\\
|G|-1 \geqslant |S \cup T| = \underbrace{\underbrace{|S|}_{\geqslant\frac{|G|}{2}} + \underbrace{|T|}_{\geqslant\frac{|G|}{2}}}_{\geqslant |G|} - |S \cap T| \Rightarrow |S\cap T| \geqslant 1 \Rightarrow S \cap T \neq \emptyset \\
\exists v_i ~:~ v_i \in S \cap T 
\end{math}
Sprzeczność istnieje cykl Hamiltona $v_1\dots v_iv_n\dots v_{i+1}v_1$
\end{proof}
\end{tw}

\subsection{Problem Komiwojażera}
\subsubsection{Dane} Sieć $S = (G, w)\quad w~:~E(G)\to [0, \infty)$\\
\subsubsection{Szukane} cykl Hamiltona o najmniejszej sumie wag krawędzi
\begin{wniosek}
Problem istnienia cyklu Hamiltona sprowadza się do problemu Komiwojażera\\
Dane: Czy $G$ ma cykl Hamiltona?
Rozwiązanie: Definiujemy $G' = K_{|G|}$\\
$w(e) = \left\lbrace
	\begin{array}{l l}
		1 & e\in E(G)\\
		2 & e\not\in E(G)
	\end{array}\right.
$
$G$ ma cykl Hamiltona $\Leftrightarrow$ istnieje rozwiązanie problemu komiwojażera dla $G'$ w wadze równej $|G|$
\end{wniosek}

%---------------%
%2. Wyklad      %
%	18.10.2011  %
%---------------%
\begin{algo}[Procedura $DFS$ - przeszukiwania drzewa w głąb]~\\
$T$ - drzewo skierowane od wierzchołka $v$\\
$v$ - wierzchołek
\begin{enumerate}
	\item Wypisz v
	\item Jeżeli $v1, \dots, v_p$ są następnikami $v$ w $T$ to wykonaj $DFS(T, v_i)$ dla $i = 1, \dots, p$
\end{enumerate}
%TODO Dodać przykład
\end{algo}

\begin{algo}[APK (Aproksymacyjne rozwiązanie problemu komiwojażera)]~\\\label{algo:apk}
Algorytm znajduje przybliżone rozwiązanie problemu komiwojażera
\begin{enumerate}
	\item Wybierz wierzchołek startowy
	\item Skonstruuj minimalne drzewo rozpinające $T$ grafu $G$
	\item Wykonaj procedurę $DFS(T, v)$
\end{enumerate}
%TODO Dodać przykład
\end{algo}

\begin{tw}\egz
Algorytm \hyperref[algo:apk]{APK} konstruuje cykl Hamiltona o co najwyżej 2-krotnie większej sumie wag krawędzi w porównaniu z optymalnym o ile jest spełniony warunek trójkąta\footnote{$\forall x,y,z \quad w(x,y) + w(y,z) \geqslant w(x,z)$}
\begin{proof}~\\
$\begin{array}{l l}
T & \text{drzewo rozpinające skonstruowane w algorytmie}\\
H^* & \text{optymalny cykl Hamiltona}\\
H & \text{cykl Hamiltona skonstruowany przez APK}
\end{array}$\\
Weźmy $e\in E(H^*)$. $H^*\smallsetminus e$ to ścieżka rozpinająca, czyli drzewo rozpinające.
$$ \frac 12w(H) \leqslant w(T) \leqslant w(H^*\smallsetminus e) \leqslant w(H^*) $$
Procedura $DFS$ przechodzi przez każdą krawędź dwa razy. Gdy komiwojażer przechodził po $T$ to w sumie przeszedł by drogę długości $2w(T)$. Pomijając odwiedzone wcześniej wierzchołki nie zwiększymy sumy wag krawędzi dzięki nierówności trójkąta.
$$ w(H) \leqslant 2w(T) \leqslant 2w(H^*) $$
\end{proof}
\end{tw}

\section{Grafy dwudzielne}
\begin{df}
Graf $G=G(V, E)$ nazywamy grafem dwudzielnym, jeśli istnieją zbiory $X, Y$ takie że $X \cup Y = V$, $X \cap Y = \emptyset$ i $\forall e\in E ~~ |e\cap X| = 1$
\end{df}

\begin{df}
Graf dwudzielny $G$ nazywamy dwudzielnym pełnym jeżeli $\forall x\in X ~\forall y\in Y ~~ xy\in E$. Jeżeli $|Y| = m$, $|X| = n$ to $G$ nazywamy $K_{n,m}$
\end{df}

\begin{tw}\egz
Graf $G$ jest dwudzielny $\Leftrightarrow$ $G$ nie zawiera nieparzystych cykli\footnote{cykli o nieparzystej długości}
\begin{proof}$\Rightarrow$
$G$ jest dwudzielny. Przypuśćmy że w $G$ jest cykl nieparzystej długości $C ~:~ x_1,x_2,\dots, x_{2k+1}$. Bez straty ogólności zakładamy, że $x_1 \in X \Rightarrow x_2\in Y \Rightarrow x_2\in X \Rightarrow \dots \Rightarrow X_{2k}\in Y \Rightarrow x_{2k+1}\in X$. \\
$x_1,x_{2k+1}\in X$ sprzeczność z dwudzielnością $G$ 
\end{proof}
\begin{proof}$\Leftarrow$\\
$G$ nie zawiera nieparzystych cykli. Pokażemy że jest dwudzielny\\
$d(x,y)$ -- długość krawędzi najkrótszej $x-y$-drogi $(x,y \in V(G))$\\
Bez utraty ogólności możemy przyjąć, że $G$ jest spójny, gdyż składowe można rozpatrzyć osobno.\\
Niech
\begin{tabular}{l l}
	$u$ & będzie dowolnym wierzchołkiem\\
	$X$ & $\{ v\in V ~:~ ^2|d(u,v)\}$\\
	$Y$ & $\{ v\in V ~:~ ^2\nmid d(u,v)\}$
\end{tabular}
czyli $X\cap Y = \emptyset ~\wedge~ X\cup Y = V(G)$\\
Pokażemy, że $\forall e\in E(G) ~~ |e\cap X| = 1$\\
Przypuśćmy, że istnieje krawędź $e = xy$ taka  że $e\subset X$\footnote{analogicznie gdy $e\subset Y$}\\
$P$ -- najkrótsza $x-u$ droga\\
$Q$ -- najkrótsza $y-u$ droga\\
%TODO Dodać obrazek
Niech $v$ będzie wspólnym wierzchołkiem $P$ i $Q$ najbliżej $x$. Odcinki dróg $P$ i $Q$ z $u$ do $v$ są takiej samej długości bo w przeciwnym przypadku jedna z nich nie była by najkrótsza. Drogi $P$ i $Q$ mają tę samą parzystość, zatem odcinki dróg $P_x, Q_y$ z $v$ do $x$ i $y$ odpowiednio mają tę samą parzystość i długość. \\Zatem cykl $xP_xvQ_yyx$ jest nieparzystej długości.
\end{proof}
\end{tw}

\begin{df}
Skojarzenie to graf którego każda składowa ma jedną krawędź.
\end{df}
\begin{df}
$G$-graf, $M$-podgraf $G$ nazywamy skojarzeniem doskonałym jeżeli jest skojarzeniem i $V(M)=V(G)$\\
Takie skojarzenie nie zawsze istnieje
\end{df}
\begin{df}
Skojarzenie $M$ nazywamy 
\begin{enumerate}
	\item maksymalnym -- jeśli nie jest zawarte w skojarzeniu o większej liczbie krawędzi
	\item maksymalnej liczności -- jeśli nie istnieje skojarzenie o większej liczbie krawędzi
\end{enumerate}
\end{df}

%---------------%
%3. Wyklad      %
%	25.10.2011  %
%---------------%
\section{Kolorowanie krawędzi}
\begin{df}
k-pokolorowaniem krawędzi grafu $G$ nazywamy funkcję $f~:~ E(G) \to G$ gdzie $|G| = k$
\end{df}

\begin{df}
k-pokolorowanie nazwiemy dobrym, jeśli żadne dwie krawędzie o wspólnym końcu nie mają tego samego koloru.
\begin{wniosek}
Dobre k-pokolorowanie krawędzie grafu $G$ definiuje podział zbioru $E(G)$ na zbiory $E_1,\dots,E_k$ takie że $G$ indukowany przez zbiór krawędzie $G[E_i]$ jest skojarzeniem.
\end{wniosek}
\begin{uwaga}
Dowolny graf $G$ ma dobre $e(G)$-pokolorowanie krawędzi
\end{uwaga}
\begin{uwaga}
Jeśli $G$ ma dobre k-pokolorowanie i $l\geqslant k$ to $G$ ma dobre l-pokolorowanie
\end{uwaga}
\end{df}

\begin{df}
Indeksem chromatycznym grafu $G$ nazywamy najmniejsze $k\in\mathbb{N}$ takie że istnieje dobre k-pokolorowanie krawędzi $G$ i oznaczamy $\chi'(G)$
%TODO Dodać przykłady
\begin{uwaga}
$\chi'(G)$ jest co najmniej $\geqslant \Delta(G)$
\end{uwaga}
\end{df}

\subsection{Problem planu zajęć}
\subsubsection{Dane}
\begin{tabular}{l l l}
	n & -- grup studenckich & $X_1,X_2,\dots,X_n$\\
	m & -- nauczycieli & $Y_1,Y_2,\dots,Y_m$\\
	$a_{ij}$ & -- liczba godzin które nauczyciel $Y_i$ ma przeprowadzić grupie $X_i$ & \\
\end{tabular}

\emph{Ułożyć plan zajęć tak żeby ostatnia godzin kończyła się najwcześniej}
\subsubsection{Założenia}
\begin{enumerate}
	\item Nie ograniczona liczba sal
	\item Żaden nauczyciel nie prowadzi dwóch zajęć jednocześnie
	\item żadna grupa nie ma dwóch zajęć jednocześnie
\end{enumerate}
\subsubsection{Rozwiązanie}
Tworzymy multigraf o zbiorze wierzchołków $X_1,X_2,\dots,X_n, Y_1,Y_2,\dots,Y_m$ wierzchołki $X_jY_i$ są połączone krawędziami $a_{ij}$\\
Dobre pokolorowanie krawędzi grafu $G$ definiuje plan zajęć.\\
Jeśli krawędź $X_jY_i$ jest koloru $c\in \{1,\dots ,k\}$ to $Y_i$ na godzinie numer $c$ ma zajęcia z $x_j$\\
\begin{przyklad}~\\
$
\begin{array}{c c c c c c}
		& Y_1 & Y_2 & Y_3 & Y_4 & Y_5 \\
	X_1 & 2 & 0 & 1 & 1 & 0\\ 
	X_2 & 0 & 1 & 0 & 1 & 0\\
	X_3 & 0 & 1 & 1 & 1 & 0\\
	X_4 & 0 & 0 & 0 & 1 & 1\\
\end{array}
$%TODO Dodać graf
~~$
\begin{array}{ccccc}
	  & X_1 & X_2 & X_3 & X_4\\
	1 & Y_1 & Y_2 & Y_4 & Y_5\\ 
	2 & Y_1 & Y_4 & Y_2 &  - \\
	3 & Y_3 &  -  &  -  & Y_4\\
	4 & Y_4 &  -  & Y_3 &  - \\
\end{array}
$
\end{przyklad}

\begin{lemat}
Niech $G$ będzie grafem spójnym, który nie jest nieparzystym cyklem. Wtedy $G$ ma 2-pokolorowanie krawędzi, takie że każdy wierzchołek stopnia $\geqslant 2$ jest końcem krawędzi w obydwu kolorach.
\begin{proof}
\begin{proof}[Przypadek 1]$G$ ma obwód Eulera\\
\begin{enumerate}
	\item Jest cyklem oczywiście parzystej długości, kolorujemy krawędzie cyklu na zmianę.
	\item $G$ nie jest cyklem zatem $G$ ma wiezchołek $v_0$ taki że $\deg v_0 \geqslant 4$\label{enum:1}
\end{enumerate}
\emph{Rozważmy obwód Eulera o początku i końcu w $v_0$, ($v_0e_1v_1\dots v_m$)}\\
Niech $\begin{array}{l l}
  E_1 & = \{e_i : 2\nmid i\}\\
  E_0 &= \{e_i : 2\mid i\}
\end{array}$\\
Wchodząc do wierzchołka $v_p$ krawędzią z $E_i$ wychodzimy krawędzią z $E_{1-i} \quad i\in\{0,1\}$
\begin{uwaga}
Przez $v_0$ też tak przechodzimy
\end{uwaga}
\end{proof}

\begin{proof}[Przypadek 2] $G$ nie ma obwodu Eulera\\
$N$ -- zbiór wierzchołków nieparzystego stopnia\\
$\overline{G} = G + \{v\} + \{vx : x\in N\}$\footnote{łączymy go ze wszystkimi wierzchołkami nieparzystego stopnia z $G$}\\
Teraz w $\overline{G}
\begin{array}{l c l}
	2\mid_{\deg_Gx} & \text{dla} & x\in N\\
	2\mid_{\deg_Gx} & \text{dla} & x\notin N\\
	2\mid_{\deg_Gv} & \text{bo}  & 2\mid_{|N|} \text{ (zgodnie z lematem o uściskach dłoni)}
\end{array}$\\
$\overline{G}$ ma obwód Eulera. Idąc po tym obwodzie kolorujemy krawędzie na zmianę tak jak w przypadku \ref{enum:1}. Usuwamy krawędzie $xv$ dla $x\in N$. \emph{Wierchołek $u\in V(G)\smallsetminus N$ w $G$ ma krawędzie każdego koloru}
Obwód Eulera przechodzi przez $u$ dwa razy, zatem ma po dwie krawędzie w każdym kolorze w $\overline{G}$. Czyli przynajmniej po jednej krawędzi w każdym kolorze w $G$.
\end{proof}
\end{proof}
\end{lemat}

\begin{lemat}
k-pokolorowanie jest dobre $\Leftrightarrow ~ \forall c\in V(G) \quad l_c(v)\footnote{\text{liczba kolorów użytych do pokolorowanie krawędzi incydentnych z }v} = \deg_Gv$
\end{lemat}
\begin{df}
k-pokolorowanie $\overline{c}$ krawędzi $G$ jest lepsze niż k-pokolorowanie $c$ krawędzi $G$ jeśli
$\sum\limits_{v\in V(G)}l_{\overline{c}}(v) > \sum\limits_{v\in V(G)}l_c $
\end{df}
\begin{df}
k-pokolorowanie krawędzi $G$ jest optymalne jeśli nie istnieje od niego lepsze.
\end{df}

%TODO Znaleźć barkujący wykład

\section{Kolorowanie wierzchołków}
%---------------%
%4. Wyklad      %
%	22.11.2011  %
%---------------%
\begin{df}
k-pokolorowaniem wierzchołków grafu $G$ nazywamy funkcję $c:~V(G)\to C$, gdzie $|C|=k$
\end{df}

\begin{df}
k-pokolorowanie nazywamy dobrym (właściwy, legalnym) jeżeli każde dwa sąsiednie wierzchołki mają inne kolory
$$ \forall x,y \in V \quad xy\in E(G) \Rightarrow c(x) \neq c(y) $$
\end{df}

\begin{df}
Podzbiór $U\subset V(G)$ nazywamy niezależnym jeżeli $\forall x,y \in U \quad xy\not\in E(G)$
\end{df}

\begin{df}
Liczba chromatyczna grafu $G$ jest to najmniejsze k, takie że istnieje dobre k-pokolorowanie.\\
$\begin{matrix}
\chi(C_{2n}) = 2 & \chi(C_{2n+1} & \chi(K_n) = n & \chi(K_{n,m}) = 2 
\chi(G) = 1 \Leftrightarrow e(G) = 0 & \chi(G) = 2 \Leftrightarrow G\text{ jest grafem dwudzielnym}
\end{matrix}$
\end{df}

\begin{df}
Graf $G$ jest krytyczny jeśli $\chi(G) > \chi(H)$ dla każdego istotnego podgrafu $H$ grafu $G$
\end{df}

\begin{uwaga}
Podgraf istotny nie jest całym grafem
\end{uwaga}

\begin{df}
Graf $G$ jest  k-krytyczny gdy jest krytyczny i $\chi(G) = k$

\begin{przyklad}
$G$ jest 3-krytyczny $\Leftrightarrow ~G$ jest nieparzystym cyklem 
\end{przyklad}
\begin{przyklad}
$K_n$ jest $n$ krytyczny
\end{przyklad}

\begin{lemat}
Jeżeli $\chi(G)=k$ to $G$ zawiera podgraf k-krytyczny
\begin{proof}
Jeśli $G$ jest krytyczny to OK. W przeciwnym razie istnieje wierzchołek lub krawędź, których usunięcie nie sprawi zmniejszenia liczby chromatycznej. Usuwamy wierzchołki i krawędzie o tej własności. dopóki się da. W wyniku tej operacji otrzymujemy podgraf k-krytyczny.
\end{proof}
\end{lemat}
\end{df}

\begin{tw}
Jeżeli graf $G$ jest k-krytyczny to $\delta(G) \geqslant k-1$
\begin{proof}
Przypuśćmy że tak nie jest\\
Przypuśćmy, że $G$ jest k-krytyczny i $\delta (G) < k-1$. Czyli $\exists v\in V(G) ~~ \deg_GV \leqslant k-2$ i krytyczności $\chi(G\smallsetminus v)\leqslant k-1$\\
Niech $c$ będzie $k-1$ pokolorowaniem $G\smallsetminus v$. Rozszerzamy je na wierzchołek $v$ przypisując najmniejszy kolor $\{c(u) : u\in N(v) \}$.\\
Ponieważ $\{c(u) : u\in N(v) \} \leqslant \deg_GV \leqslant k-2$ to przypisujemy wierzchołkowi $v$ kolor najwyżej $k-1$ otrzymujemy pokolorowanie $G$ na $k-1$ kolorów wbrew $\chi(G)=k$
\end{proof}

\begin{wniosek}
Jeśli $G$ jest k-krytyczne to zawiera co najmniej $k$ wierzchołków stopnia co najmniej $k-1$
\end{wniosek}
\begin{proof}
Niech $v$ będzie wierzchołkiem takim że $\deg_GV = \delta \delta(G)$. Z twierdzenia $\delta(G) \geqslant k-1$ zatem $|\mathbf{N}[v]| \geqslant k$ i $\forall u\in\mathbf{N}[v] ~~ \deg_Gu\geqslant k-1$ \footnote{sąsiedztwo domknięte $\mathbf{N} = N(v) \cup \{v\}$ }
\end{proof}

\begin{wniosek}
$\chi(G) \leqslant \Delta(G)+1$ dla dowolnego grafu
\begin{proof}
Weźmy graf i pokolorujmy jego wierzchołki po koli jak najmniejszym kolorem. Każdy wierzchołek mam $\Delta(G)$ zabronionych kolorów więc robimy kolorów o $1$ więcej.
\end{proof}
\end{wniosek}
\end{tw}

\begin{tw}[ \href{http://en.wikipedia.org/wiki/R._Leonard_Brooks}{Brooks'a}]
Jeżeli graf $G$ jest spójny i różny od $K_n$ i $C_{2n+1}$ dla każdego $n$ to $$\chi(G) \leqslant \Delta(G)$$
\begin{uwaga}
Liczby $\chi(G)$ i $\Delta(G)$ mogą różnić się dowolnie np. $\chi(K_{n,n}) = 2,~~ \Delta(K_{n,n}) = n$
\end{uwaga}
\begin{uwaga}
$\chi(G)\geqslant\omega(G)$\footnote{$\omega(G)$ = rozmiar największej kliki (podzbiór wierzchołków grafu pełnego)}
\end{uwaga}
\end{tw}

\begin{tw}[Descartes-Mycielski]
Dla dowolnej liczny naturalnej $k$ istnieje graf $G_k$ taki że $\chi(G_k)= k$ i $G_k$ nie zawiera trójkąta
\begin{proof}Nie wprost\\
Niech $G_k = (V_k, E_k)$ gdzie $V_k = \{v_1,\dots ,v_k\}$ będzie grafem bez trójkątów takim że $\chi(G_k) = k$\\
Definiujemy $G_{k+1} : V_{k+1} = \{v_1,\dots ,v_k, u_1,\dots, u_k, v\}$\\
$E(G_{k+1}) = E(G_k) \cup \{v_iu_j: ~ v_iv_j\in E(G_k)\} \cup \{u_iv : ~ i=\{1, \dots, n\}\}$
$G_k$ nie ma trójkątów więc dowodzimy indukcyjnie po $k$. Zakładamy, że $G_{k-1}$ nie ma trójkątów. Przypuśćmy, że $\Delta\{x,y,z\}$\\
$|\{x,y,z\} \cap \{u_1,\dots, u_n\}| \leqslant 1$ czyli $v\not\in \{x,y,z\}$ z założenia indukcyjnego $\{x,y,z\} \not\subset \{v_1,\dots, v_n\}$\\
Pozostaje jedyna możliwość $\begin{matrix}x\in & \{u_1,\dots,u_n\} \\ y,z\in & \{v_1,\dots,v_n\}\end{matrix}$. Przyjmujemy, że $\begin{matrix}x = u_i \\ y = v_j \\ z = v_k\end{matrix}$\\
$u_i,v_j,v_k$ tworzą trójkąt wbrew założeniu indukcyjnemu. Sprzeczność.\\
\emph{Teraz pokażemy, że $\chi(G_k) = k$\\}
\textbf{Założenie indukcyjne:} $\chi(G_{k-1} = k-1$\\
\textbf{Teza:} $\chi(G_k) = k$\\
Rozważmy graf $G_k~~ V(G)=\{v_1,\dots ,v_k, u_1,\dots, u_k, v\}$. Załóżmy, że $G_k[\{v_1,\dots,v_n\}]$ jest izomorficzny $G_{k-1}$. Z założenia indukcyjnego istnieje (k-1)-pokolorowanie $G_k[\{v_1,\dots,v_n\}]$ nazwijmy je $C^\#$\\
Definiujemy k-pokolorowanie $G_k$\\
$C(x) = \left\{\begin{matrix}
C^\# & \text{jeśli } x\in\{v_1,\dots,v_n\}\\
C(v_i) & \text{jeśli } x=u_i\\
k & \text{jeśli } x=v
\end{matrix}\right.$\\
Pokazaliśmy, że $\chi(G_k) \leqslant k$. Przypuśćmy, że istnieje (k-1)-pokolorowanie grafu $G_K{k+1}$, nazwijmy je $\xi$. Zauważmy że żaden z wierzchołków $u_1,\dots,u_n$ nie jest pokolorowany kolorem $\xi(v)$.\\
Definiujemy nowe pokolorowanie grafu indukowanego $V$ czyli $G_k[\{v_1,\dots,v_n\} \cong G_{k-1}\quad \phi(v_i) = 
\left\{\begin{matrix}
	\xi(v_i) & \text{jeśli } \xi(v_i) \neq \xi(v)\\
	\xi(u_i) & \text{jeśli } \xi(v_i) = \xi(v)
\end{matrix}\right.$
Pokolorowanie $\xi$ jest dobrym (k-2)-pokolorowaniem wbrew założeniu indukcyjnemu. Sprzeczność
%TODO Dodać rysunki
\end{proof}
\end{tw}

%---------------%
%5. Wyklad      %
%	06.12.2011  %
%---------------%

\begin{problem}
Fabryka produkuje $n$ substancji chemicznych $v_1m\dots,v_n$. Niektóre nie powinny być przechowywane w tych samych magazynach. Ile potrzebujemy najmniej magazynów, aby każda substancja była w jakimś magazynie ale z żadną substancją z którą nie powinna być?\\
Rozwiązaniem jest liczba chromatyczna następującego grafu:
$\begin{array}{l l}
V = \{v_1,\dots,v_n\} \\ 
E =  \{v_iv_j : ~ \text{substancje } v_iv_j \text{ nie powinny być razem w magazynie}\}
\end{array}$
\end{problem}

\section{(Multi)Grafy planarne (płaskie)}
\begin{df}
Grafem planarnym (płaskim) nazywamy graf który można narysować na płaszczyźnie tak, aby krawędzie przecinały się jedynie w wierzchołkach.
\end{df}
%TODO Dodać przykłady

\begin{tw}
Graf można narysować na płaszczyźnie $\Leftrightarrow$ można go narysować na sferze
%TODO Dodać dowód
\end{tw}

\begin{df}
Reprezentacją płaską multigrafu planarnego nazywamy rysunek grafu\\
Reprezentacja multigrafu planarnego dzieli płaszczyznę na obszary nazywane regionami
%TODO Dodać przykład
\end{df}

\begin{df}
Mówimy że region $f$ jest incydentny z krawędzią $e$, jeżeli $e$ leży w całości na brzegu $f$
\end{df}

\begin{uwaga}
Jeżeli krawędź jest mostem to jest incydentna z jednym regionem, w przeciwnym przypadku z dwoma regionami
\end{uwaga}

\begin{df}[Graf dualny]~\\
\begin{itemize}
	\item Wierzchołkom w $G*$ odpowiadają regiony czyli $V(G*) = F(G)$.
	\item $f*g*$ jest krawędzią w $G*$ jeśli regiony $f$ i $g$ mają wspólną krawędź na brzegu
	\item $f*g*$ łączymy tyloma krawędziami ile $f$ i $g$ mają wspólnych w $G$
\end{itemize}
\end{df}

%TODO Chyba czegoś tu brakuje

%---------------%
%6. Wyklad      %
%   13.12.2011  %
%---------------%

\begin{df}
$G$ -- reprezentacja płaska grafu\\
Definiujemy graf dualny $G*$:
\begin{itemize}
	\item $V(G*) = F(G)$
	\item para $f*,g*$ jest krawędzią w $G*$ o krotności $p$ jeśli brzegi $f,g$ mają wspólnych $p$ krawędzi
\end{itemize}
\end{df}

\subsection{Własności grafów dualnych}
\begin{itemize}
	\item Dwie różne reprezentacje płaskie tego samego grafu mogą mieć nieizomorficzne grafy dualne
	\item Jeśli $G$ jest reprezentacją płaską grafu planarnego $G*$ to też jest płaski
	\item $|G*| = \Phi(G)$
	\item $e(G*) = e(G)$
	\item $\forall f\in F(G) ~~ \deg_{G*}f* = \deg_Gf$
\end{itemize}

\begin{tw}
Niech $G$ będzie reprezentacją płaską. Wtedy $$ \sum\limits_{f\in F(G)} \deg_Gf = 2e(G) $$
\begin{proof}
$ \sum\limits_{f\in F(G)} \deg_Gf = \sum\limits_{f\in F(G)} \deg_{G*}f^* = \sum\limits_{f^*\in V(G^*)} \deg_{G*}f* = 2e(G^*) = 2e(G)$
\end{proof}
\end{tw}

\begin{tw}[Eulera 1750]
Jeśli $G^*$ jest reprezentacją płaską grafu spójnego $G$ to $|G| - e(G) + \Phi(G) = 2$\marginpar{\tiny{Wzór Eulera}}
\begin{proof}
Indukcja po $\Phi(G)$\\
$\Phi(G) = 1$ w $G$ nie ma cyklu i jest spójny (jest drzewem) więc $e(G) = |G|-1$\\
$\Phi(G) = k > 1$ zakładamy że wzór Eulera jest spełniony dla wszystkich $G'$ takich że $\Phi(G') \leqslant k-1$\\
W $G$ powstaje cykl (np. krawędzie tworzące brzeg pewnej łamanej). $e$ to krawędź w $C$ która nie jest mostem więc leży na brzegu dokładnie dwóch ścian $f_1, f_2$
%TODO Dodć obrazek
Usuwamy $e$ z $G$. Dostajemy reprezentację płaską $H$ jakiegoś grafu $H$.\\
$|H| = |G| \quad \Phi(H) = \Phi(G) -1 \quad \Phi(H) \leqslant k-1$ czyli wzór Eulera jest spełniony dla $H$\\
$|G| - e(G) + \Phi(G) = |H| - e(H) - 1 + \Phi(H) + 1 = 2$
\end{proof}

\begin{wniosek}
Graf $G$ -- planarny graf prosty. Wtedy $e(G) \leqslant 3|G| -6$
\end{wniosek}
\begin{wniosek}\label{wniosek:2}
Graf $G$ -- planarny graf prosty. Wtedy $\delta(G) \leqslant 5$
\end{wniosek}
\end{tw}

\begin{df}
Pod-podziałem krawędzi $xy$ w grafie $G$ nazywamy operację polegającą ma zastępowaniu krawędzi $xy$ dwoma krawędziami $xz$ i $zy$ przy czym $z$ jest wierzchołkiem nie należącym do $V(G)$
\end{df}

\begin{df}
Graf $H$ nazywamy pod-podziałem grafu $G$ jeśli $H$ można otrzymać z $G$ przez ciąg podziałów krawędzi
\end{df}


\begin{tw}[Kuratowski 1930]
$G$ jest planarny $\Leftrightarrow~G$ nie zawiera pod-podziałów grafu $K_5$ lub $K_{3,3}$

\begin{proof}$\Rightarrow$\\
Jeżeli $G$ zawiera pod-podział $K_5$ lub $K_{3,3}$ to $G$ nie jest planarny to pod-podziałby $K_5$ lub $K_{3,3}$ nie są planarne
\end{proof}
%TODO Dodać obrazek
\begin{proof}$\Leftarrow$\\
...
Wierzchołek minimalnego stopnia umieszczamy na stosie i usuwamy z grafu, aż usuniemy wszystkie wierzchołki. Zdejmujemy wierzchołki ze stosu i kolorujemy najmniejszym kolorem niepowodującym
konfliktów. Z wniosku \ref{wniosek:2} w momencie usuwania wierzchołka to ten wierzchołek ma stopień $\leqslant 5$. Zatem gdy chcemy pokolorować wierzchołek to pokolorowanych jest $\leqslant 5$ jego sąsiadów.
\end{proof}
\end{tw}

\begin{tw}
Jeżeli $G$ jest grafem planarnym to $\chi(G) \leqslant 5$

\begin{proof}
Przypuśćmy że istnieje kontrprzykład, czyli graf planarny $G$ taki że $\chi(G) > 5$\\
Wyrzucamy wierzchołki z $G$ aż otrzymamy graf $H$ o liczbie chromatycznej $6$. Oczywiście $H$ jest planarny. Niech $F$ będzie 6-krytycznym planarnym podgrafem grafu $H$. Z wniosku \ref{wniosek:2} $\delta(F) \leqslant 5$. $F$ jest 6-krytyczny więc $\delta(F) \geqslant 6-1 = 5$ stąd $\delta(F) = 5$. Niech $X$ będzie wierzchołkiem z $F$ stopnia 5. $F$ jest 6-krytyczny, więc $\chi(F-x) = 5$. Kolorujemy $F-x$ kolorami $c_1,\dots,c_5$. Wszyscy sąsiedzi $x$ mają różne kolory bo w przeciwnym przypadku można by pokolorować $x$ jednym z kolorów $c_1,\dots,c_5$ i wtedy $F$ byłby pokolorowany na 5 kolorów. Sprzeczność.\\
Możemy tak nazwać kolory że $v_1$ ma kolor $c_1$, $v_2$ --- $c_2$, \dots $x$ ma kolor $c_5$. Dla $i,h\in \{1,\dots, 5\}, j\neq i$Niech $F_{ij}$ będzie podgrafem indukowanym przez wierzchołki o kolorach $c_i$ oraz $c_j$. 
%TODO Dodać rysunek
$v_i$ i $v_j$ należą do tej samej składowej grafu $F_{ij}$ bo w przeciwnym przypadku można by w składowej $F_{ij}$ zawierającej $v_i$ zamienić kolory $c_ic_j$, otrzymalibyśmy dobre pokolorowanie $F-x$ na 5 kolorów takie że sąsiedzi $x$ są w 4 kolorach więc można by pokolorować $H$ na 5 kolorów. Sprzeczność\\
%TODO Dodać rysunki
Istnieje droga $P_{2,4}$  łącząca $v_2$ z $v_4$ w $F_{24}$ (wierzchołki tej drogi mają kolory $c_2$  i $c_4$) Analogicznie istnieje droga $P_{13}$ z $v_1$ do $v_3$ w $F_{13}$ złożona z wierzchołków w kolorach $c_1$i $c_3$. Niech $c = xv_2P_{24}v_4x ~~ v_1$ %Tu nie mogę odczytać
Droga $P_{13}$ może przeciąć cykl $C$ w wierzchołku, bo $G$ jest planarny. To jest niemożliwe, bo z jednej strony ten wierzchołek należy do $c$ czyli ma kolor $c_2$ lub $c_4$ a z drugiej strony należy do $P_{13}$ czyli ma kolor $c_1$ lub $c_3$. Sprzeczność.
\end{proof}
\end{tw}

\subsection{Skojarzenia grafów}

\begin{df}
Podgraf $M$ grafu $G$ nazywamy skojarzeniem jeśli każda składowa $M$ jest izomorficzna z $K_2$
\end{df}

\begin{df}
Zbiór wierzchołków $C\subset V(G)$ nazywamy wierzchołkowym połączeniem $G$ jeśli każda krawędź $G$ jest incydentna z wierzchołkiem z $C$.
\end{df}

\begin{df}
Drogę zaczynającą się z wierzchołka $x\in A$ który nie należy do żadnej krawędzi z$M$ składający się na przemian z krawędzi należących do $E-E(M)$ i $E(M)$ nazywamy drogą naprzemienną względem $M$
%TODO Dodać rysunek
\end{df}

\begin{df}
Drogę naprzemienną względem $M$ zawierającą się w wierzchołku z $A$ nazywamy nienasyconą względem $M$ jeśli jej koniec należy do $B$ i nie należy do żadnej krawędzi z $M$.
%TODO Dodać rysunek
\end{df}

%---------------%
%7. Wyklad      %
%   15.12.2011  %
%---------------%

\begin{df}
$G$ - graf, $C\subset V(G)$ jest wierzchołkowym pokryciem grafu $G$ jeżeli każda krawędź ma przynajmniej jeden koniec w $C$
\end{df}

\begin{tw}[Koniga]
Rozmiar największego skojarzenia w grafie dwudzielnym jest równy liczbie wierzchołków w najmniejszym wierzchołkowym pokryciu.
\begin{proof}
Niech $\begin{matrix}
m(G) \text{ -- rozmiar najliczniejszego skojarzenia}\\
c(G) \text{ -- liczność majmniejszego pokrycia}
\end{matrix}$
\begin{enumerate}
	\item Dowodzimy że $m(G) \leqslant c(G)$\\
	Niech $M$ -- skojarzenie rozmiaru $m(G)$. Aby pokryć wszystkie krawędzie z $M$ (bo żaden wierzchołek nie należy do więcej niż jednej krawędzi).
	\item $c(G) \leqslant m(G)$\\
	Niech $M$ będzie skojarzeniem liczności $m(G)$ skonstruujmy pokrycie $C$ liczności $c(G)$. $G = (A,B,E)$ jest grafem dwudzielnym, $V(G) = A \cup B$. \\
	Z krawędzi $e\in E(M)$ do $C$ wybierzmy jej koniec należący do $B$, jeśli jest on końcem naprzemiennej ścieżki względem $M$, w przeciwnym przypadku do $C$ wybieramy koniec należący do $A$\\
	$|C| = |M| = m(G)$
	\item Pokażemy że $C$ jest pokryciem \\
	Niech $ab$ będzie dowolną krawędzią w $G$, $a\in A,~ b\in B$ 
	\begin{tabular}{l}
	

		jeśli $ab\in E(M)$ to $a\in C$ lub $b\in C$	\\
		jeśli $ab\not\in E(M)$: Zauważmy że $\exists e\in E(M) ~~ a\in e$ lub $b\in e$.
	\end{tabular}\\
	 Gdyby było inaczej to krawędź $ab$ można by dodać do $M$ otrzymując skojarzenie liczności $m(G)+1$. Sprzeczność \\
	 Jeżeli $\exists e\in E(M) ~~ b\in e $ to $b\in C$ ponieważ $e\in R(N)$ i $b$ jest końcem naprzemiennej ścieżki konkretnie ścieżki $ab$.\\
	 Jeżeli $b\not\in V(M)~ \cap ~\exists e\in E(M) \quad a\in e$ to niech $e = ab'$\\
	 Jeżeli $a\in C$ to ok $\rightarrow$ krawędź $ab$ ma koniec w $C$\\
	 Jeżeli $a\in C$ to  $b'\in C$ i $b'$ jest końcem naprzemiennej ścieżki $P$
	 \begin{itemize}
	 	\item Jeśli $P$ przechodzi przez $b$ to $b$ jest końcem pewnej naprzemiennej ścieżki (konkretnie początkowego docinka P -- czyli ta ścieżka jest krótsza od $P$) to znaczy że $b\in C$
	 	\item Jeśli $P$ nie przechodzi przez $b$ to ścieżka $P_{ab}$ jest ścieżką naprzemienną kończącą się w $b$, a $b$ nie należy do żadnej krawędzi z $M$
	 	%TODO Dodać obrazek
	 \end{itemize}
	 $b$ jest końcem nienasyconej ścieżki naprzemiennej $P'=ab$. Niech $E(M') = E(M) \div E(P')$
	 %TODO Dodać ilustrację
	 $|M'| = |M| + 1$ sprzeczność z definicją $M$ jako najliczniejszego skojarzenia
\end{enumerate}
\end{proof}
\end{tw}

\begin{tw}[\href{http://pl.wikipedia.org/wiki/Philip_Hall}{Halla} (Wersja grafowa)]\label{tw:halla}
Każdą pannę można wydać za mąż za kawalera, którego lubi $\Leftrightarrow$ każdy podzbiór $S$ panien lubi co najmniej (w sumie) $|S|$ kawalerów.
\begin{proof}
Zaręczamy pary zachłannie, każda kolejna pana zaręcza się z pierwszym wolnym kawalerem, którego lubi o ile taki istnieje. W przeciwnym przypadku urządza przyjęcie zapraszając wszystkich których lubi oni zapraszają swoje narzeczone a one kawalerów których lubią i tak dalej. To znaczy każdy kawaler przychodzi na przyjęcie z narzeczoną (o ile taką ma), a każda panna ze wszystkimi kawalerami których lubi.\\
Przypuśćmy że na przyjęciu nie ma wolnego kawalera. Niech $S$ oznacza zbiór panien na przyjęciu. Kawalerów na przyjęciu czyli wszystkich lubianych przez panny z $S$ jest $S-1$ -- narzeczeni wszystkich panien oprócz gospodyni. Sprzeczność z założeniem.\\
Kawaler bez pary tańczy z panną która go zaprosiła, jej narzeczony z panną która go zaprosiła i tak dalej, aż pewien kawaler tańczy z gospodynią. Tańczące pary zrywają swoje zaręczyny o zaręczają się z partnerem od tańca. W ten sposób kolejne panny możemy nareszcie zaręczyć. Gdy już nie ma wolnych panien urządzamy ślub. 
\end{proof}
\end{tw}

\begin{tw}[\href{http://pl.wikipedia.org/wiki/Philip_Hall}{Halla} 1935]
$G=(X,Y,Z)$ graf dwudzielny. \\Jeśli $\forall S\subset X\quad |N(S)| \geqslant |S|$ to w $G$ istnieje skojarzenie liczności $|X|~$ \footnote{$N(S) = \{y\in Y: ~ \exists x\in X ~~ yx \in E\}$}
\end{tw}

%---------------%
%8. Wyklad      %
%   20.12.2011  %
%---------------%
\subsection{Systemy reprezentantów}
\begin{df}
Niech $(A_1, \dots , A_m)$ - ciąg podzbiorów zbioru $X$\\
Ciąg $(a_1, \dots , a_n)$ nazywamy systemem różnych reprezentantów ciągu $(A_1,\dots, A_m)$ jeśli 
\begin{enumerate}
	\item $\forall i\in\{i,\dots , n\} ~ a_i \in A_i$
	\item $\forall i,j\in\{1, \dots ,n\} ~ i\neq j ~\Rightarrow ~a_i \neq a_j$
\end{enumerate}
\end{df}

\begin{tw}[\href{http://pl.wikipedia.org/wiki/Philip_Hall}{Halla} (Wersja transwersalowa\footnote{transwersala -- system reprezentantów})]~\\
Ciąg zbiorów $(A_1, dots , A_m)$  ma system różnych reprezentantów $\Leftrightarrow ~ \forall y\in\{1,\dots ,n\} ~~ |y| \leqslant |\bigcup\limits_{i\in y} A_i|$
\begin{proof}~\\
Definiujemy graf dwudzielny $G=(X,Y,E) \quad x = \{A_1, \dots , A_m\} ~ Y = \bigcup\limits_{i=1}^n A_i $ i powiemy że $A_iy_i\in E \Leftrightarrow y_i \in A_i$\\
Skojarzenie $A_1y_1 \dots A_ny_n$ definiuje system różnych reprezentantów z twierdzenia \ref{tw:halla} wynika że istnieje skojarzenie $\Leftrightarrow ~ S\subset \{A_1, \dots , A_m\} \quad |N(S)| \geqslant |S| = y$
\end{proof}
\end{tw}

\begin{problem}
\begin{center}
Czy można przydzielić pracę pracownikom, tak aby każda była wykonana przez pracownika który ją umie?\\
\end{center}
Zakładamy że każda praca jest wykonana przez jednego pracownika i każdy pracownik wykonuje jedną pracę
Problem sprowadza się do znalezienia systemu różnych reprezentantów.
\end{problem}

\subsection{Maksymalny przepływ w sieci}
\begin{df}
Parę $S = (G,C)$ gdzie $G=(V,E)$ jest grafem zorientowanym, a $c: ~E\to \mathbb{R}$ (nazywamy funkcją przepustowości) nazywamy siecią.
\end{df}

\begin{df}
Niech $s,t$ będą wyróżnionymi wierzchołkami sieci $S$. Przepływem z $s$ do $t$ nazywamy dowolną funkcję $f: ~ E-to \mathbb{R_+} \cup \{0\}$ taką że:
\begin{enumerate}
	\item $\forall e\in E \quad f(e) \leqslant c(e)$
	\item $D_iY_e(v) := \sum\limits_{u: vu\in E} f(vu) - \sum\limits_{u: uv\in E}f(uv) = 0 $ dla każdego $v\in V \smallsetminus \{s, t\}$ 
\end{enumerate}
Wielkość $w(f) = D_iY_f(s)$ nazywamy wartością przepływu
\end{df}

\begin{df}
Przekrojem $P(A)$ odpowiadającym niepustemu zbiorowi $A\subset V$ nazywamy zbiór krawędzi $P(A) = E \cap A \times (V \smallsetminus A)$ (jeden koniec w $A$ a drugi nie)
\end{df}

\begin{df}
Dla dowolnego przepływu $f$ w sieci $S$ definiujemy przepływ przez przekrój $P(A)$ jako $f(A, V\smallsetminus A) = \sum\limits_{e\in P(A)}f(e)$
\end{df}

%TODO Przykłąd

\begin{lemat}
Jeśli $s\in A$, $t\not\in A$ to dla dowolnego przepływu z $s$ do $t$ 
$$w(f) = f(A,V\smallsetminus A) - f(V\smallsetminus A, A)$$
\begin{proof}
$$w(f) = Div_f(s) = Div_f(s) + 0 = 
\sum\limits_{v\in A} Div_f(v) 
= \sum\limits_{v\in A}\left(\sum\limits_{u: uv\in E}f(uv) - \sum\limits_{u: vu\in E}f(vu)\right)
= f(A, V\smallsetminus A ) - f(V\smallsetminus A, A)$$
\end{proof}
\end{lemat}

\begin{df}
Przepustowością przekroju $P(a)$ nazywamy liczbę $c(A, V \smallsetminus A) = \sum\limits_{e\in P(A)} c(e)$
\end{df}

\begin{df}
Minimalny przekrój między $s$ i $t$ to jest przekrój $P(A)$ taki że $s\in A$, $t\not\in A$ o minimalnej przepustowości
\end{df}

\begin{tw}[\href{http://pl.wikipedia.org/wiki/Metoda_Forda-Fulkersona}{Forda-Fulkersona}]
Wartość każdego przepływu z $s$ do $t$ nie przekracza przepustowości minimalnego przekroju między $s$ i $t$, przy czym istnieje przepływ osiągający tą wartość
\begin{proof}
Niech $P(A)$ będzie przekrojem minimalnym 
$$w)f_ = f(A, V\smallsetminus A) - f(V\smallsetminus A, A) \leqslant f(A, V\smallsetminus A) = \sum\limits_{e\in P(A)} f(e) \leqslant \sum\limits_{e\in P(A)} c(e) = c(A, V\smallsetminus A) $$
\end{proof}
\end{tw}

\begin{df}
Krawędź $e$ sieci $S$ jest użyteczna z $u$ do $v$ jeśli $e=uv$ i $f(e) < c(e)$ lub jeśli $e=vu$ i $f(e) > 0$ (jeśli można zwiększyć przepływ)
\end{df}

\begin{df} 
Ścieżką rozszerzającą z $s$ do $t$ dla danego przepływu $f$ to ciąg $v_0e_1v_1\dots e_lv_l$ taki że $v_0 = s, ~ v_l=t$ oraz $\forall i\in \{1,\dots , l\}$ krawędź $e_i$ jest użyteczna z $v_{i-1}$ do $v_i$ względem przepływu $f$.
\end{df}

%---------------%
%9. Wyklad      %
%   03.01.2012  %
%---------------%

%TODO Przykład

\begin{tw}
Następujące warunki są równoważne:
\begin{enumerate}
	\item Przepływ $f$ z $s$ do $t$ jest maksymalny\label{warunek:1}
	\item Nie istnieje ścieżka rozszerzająca z $s$ do $t$ względem $f$
	\item $w(f) = c(A, V \smallsetminus A)$ dla pewnego $A\subset V\smallsetminus \{t\}~~s\in A$
\end{enumerate}
\begin{proof}
\begin{itemize}
	\item $1\Rightarrow 2$\\
	Zachodzi ponieważ gdyby istniała ścieżka rozszerzająca względem $f$ to można by było znaleźć większy przepływ, a ten jest już maksymalny
	\item $2\Rightarrow 3$\\
	Zakładamy że nie istnieje ścieżka rozszerzająca z $s$ do $t$ względem $f$. Niech $A$ będzie zbiorem wierzchołków $V$ takim że istnieje ścieżka rozszerzająca z $s$ do $v$ względem $v$. Zauważmy że:
	\begin{enumerate}
		\item $t\not\in A$
		\item $e = vu ~~ v\in A \wedge u\not\in A$ to $f(e) = c(e)$ bo gdyby $f(e) <c(e)$ to istniałaby ścieżka rozszerzająca z $s$ do $u$ i wtedy $u\in A$ wbrew założeniu
		\item $e = uv ~~ v\in A \wedge u\not\in A$ to $f(e) = 0$ bo gdyby $f(e) > 0 $ to $u\in A$ wbrew założeniu
	\end{enumerate}
	$$w(f) = \sum\limits_{e\in P(A, V\smallsetminus A)}f(e) - \sum\limits_{e\in P(V \smallsetminus A, A)}f(e) = \sum\limits_{e\in P(A, V\smallsetminus A)}c(e) - 0 = c(A, V\smallsetminus A)$$
	\item $3 \Rightarrow 1$\\
	$f$ jest maksymalny bo wartość przepływu nie może przekroczyć żadnej wartości przekroju
\end{itemize}
\end{proof}
\end{tw}

%TODO Dodać przykład i algorytm

%---------------%
%9. Wyklad      %
%   10.01.2012  %
%---------------%

\subsection{Matroidy}
\begin{df}
Matroidem nazywamy parę $M = (E,C)$ gdzie $E$ jest zbiorem skończonym a $C\subset Z^E$
\begin{enumerate}
	\item $\emptyset\in C$ i $\forall A,B\subset E~~ A\subset B \wedge B\in C \Rightarrow A\in C$
	\item $\forall A,B\in C ~~ |B| = |A| +1 \Rightarrow \exists e\in B\smallsetminus A ~~ A\cup \{e\} \in C$
\end{enumerate}
Zbiory należące do $C$ nazywamy zbiorami niezależnymi
\end{df}
\begin{przyklad}
$E$ -- zbiór skończony, $C = Z^E$
\end{przyklad}
\begin{przyklad}
Matroid podziałowy\\
$E$ -- zbiór skończony, $D_1, \dots , D_k$ -- podział $E$\\
$C = \{A\subset E: ~~ |A\cap D_i| \leqslant 1\}$
\end{przyklad}


%TODO Dodać więcej przykładów

\begin{tw}
Niech $C\subset Z^E$ spełnia warunek \ref{warunek:1} wtedy para $(E,C)$ jest matroidem $\Leftrightarrow$ dla dowolnego $D\subset E$ każde dwa maksymalne w $D$ zbiory niezależne są tej samej liczności
\begin{proof}
$\Rightarrow$ Przypuśćmy że istnieją zbiory niezależne $A,B$ maksymalne w pewnym $D\subset E ~~ |B| > |A|$, z
\begin{enumerate}
	\item $\Rightarrow ~~ \exists B'\subset B$ takie  że $B'\in C ~~ |B'| = |A| + 1$
	\item $\Rightarrow ~~ \exists e\in B'\smallsetminus A \subset D \quad A\cup \{e\} \in C$
\end{enumerate}
czyli nie jest zbiorem niezależnym maksymalnym w $D$\\
$\Leftarrow$ Chcemy pokazać że prawdą jest $2$\\
Rozważmy $A,B\in C ~~ |B| = |A|+1 ~~ |D+A\cup B$. Przypuśćmy że $2$ nie zachodzi, czyli $\forall e\in B'\smallsetminus A~~ A\cup\{e\}\in C \Leftrightarrow ~A$ jest maksymalnym zbiorem niezależnym. $B$ jest zawarty w pewnym zbiorze niezależnym, maksymalnym w $D$. $B'| \geqslant |B| > |A|$ sprzeczność z $3$
\end{proof}
%TODO Uporządkować ten dowód do czego się ona tak naprawdę odnosi
\end{tw}

\begin{wniosek}
Wszystkie zbiory niezależne, maksymalne w $E$ matroidu $(E,C)$ mają taką samą liczność
\end{wniosek}

\begin{tw}[\href{http://www.wmie.uz.zgora.pl/~edrgas/pliki/material\%20dla\%20zao\%20matroidy.pdf}{Rado-Edmondsa 1971}]
\end{tw}


\end{document}