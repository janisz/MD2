\documentclass[12pt,a4paper]{article}
\usepackage[polish]{babel}
\usepackage[utf8]{inputenc}
\usepackage[T1]{fontenc}
\usepackage[a4paper]{geometry}
\usepackage{fullpage}
\usepackage{pdflscape}
\usepackage{float}

\usepackage{amsthm}
\usepackage{amsfonts}
\usepackage{amsmath} %pakiet potrzebny do \eqref
\usepackage{amssymb} %pakiet potrzebny do \blacksquare i \mathbb{•}


\usepackage{graphicx}

\newtheorem{lemat}{Lemat}
\newtheorem{tw}{Twierdzenie}
\newtheorem{przyklad}{Przykład}
\theoremstyle{definition}
\newtheorem{df}{Definicja}

\usepackage[pdfauthor={Janisz},%
pdftitle={MD2},%
pagebackref=true,%
pdftex]{hyperref}
\hypersetup{colorlinks=false}


\pagestyle{plain}
\begin{document}
\title{ Matematyka Dyskretna 2}
\author{Tomasz Janiszewski}
\date{\today}
\maketitle

\tableofcontents

\DeclareGraphicsExtensions{.pdf,.png,.jpg}
\begin{center}
\leavevmode
\includegraphics[width=1 in]{../img/by-sa.png}
\end{center}
\label{fig:cc}
%insert a link to the licence and its description below
\scriptsize{Ten utwór jest dostępny na licencji  \href{http://creativecommons.org/licenses/by-sa/3.0/pl/}{Creative Commons Uznanie autorstwa-Na tych samych warunkach 3.0 Polska.}}

\pagebreak
%1. Wyklad
%	4.10.2011
\section{Ścieżka i cykl Eulera\footnote{Leonhard Euler (ur. 15 kwietnia 1707 w Bazylei, zm. 18 września 1783 w Petersburgu) – szwajcarski matematyk i fizyk; był pionierem w wielu obszarach obu tych nauk. Większą część życia spędził w Rosji i Prusach. Jest uważany za jednego z najbardziej produktywnych matematyków w historii}}
%TODO Dodać opis problemu
\begin{df}
Ścieżką Eulera w grafie $G$ nazywamy ścieżkę $v_0e_1v_1e_2 \dots e_nv_n$ taką że każda krawędź występuje na niej tylko raz
\end{df}
\begin{df}[Obwód Eulera]~\\
Jeżeli w danej $v_0 = v_n$ to ścieżkę nazywamy ścieżką zamkniętą lub obwodem Eulera
\end{df}
\begin{df}[Cykl Eulera]~\\
Jeżeli wierzchołki  w obwodem Eulera się nie powtarzają to taki obwód nazywamy cyklem Eulera.
\end{df}

\begin{lemat}\label{lemat:1}
Jeżeli $G$ jest grafem takim że $\delta (G) \geqslant 2$ to $G$ posiada cykl
\begin{proof}
Rozważmy najdłuższą drogę (wierzchołki się nie powtarzają) $v_0e_1v_1 \dots e_kv_k$ oraz niech $\deg v_0 \geqslant \delta(G) \geqslant 2$. Niech sąsiad $v_0 \neq v_1$ nazywa się $a$. $a \not\in \{v_2, \dots, v_k\}$ bo gdyby należał do ścieżki to $av_0v_1 \dots v_k$ byłaby dłuższa niż $v_0 \dots v_k$ wbrew założeniu. Czyli $a\in \{v_2, \dots, v_k\}$ czyli $a = v_i$.\\ Mamy cykl! 
%TODO Dodać ilustrację
\end{proof}
\end{lemat}

\begin{tw}[Eulera]~\\
Graf spójny ma obwód Eulera $\Leftrightarrow$ wszystkie wierzchołki mają parzyste stopnie
\begin{proof}
$\Rightarrow$\\
Przemierzając obwód Eulera wchodzimy i wychodzimy do każdego wierzchołka tyle samo razy za każdym razem inną krawędzią. Ponieważ każda krawędź leży w obwodzie Eulera to każdy wierzchołek jest parzystego stopnia
\end{proof}
\begin{proof}
$\Leftarrow$ \emph{nie wprost}\\
Przypuśćmy że twierdzenie nie jest prawdziwe. Ze wszystkich kontrprzykładów weźmy ten który ma najmniej krawędzi, nazwijmy go $G$. To znaczy $G$ jest spójny, wszystkie wierzchołki ma parzystego stopnia a nie ma obwodu Eulera. To znaczy że $|G| > 1$\footnote{graf o jednym wierzchołku ma obwód Eulera} oraz żaden wierzchołek nie jest stopnia 0 lub 1, każdy jest co stopnia co najmniej 2. Czyli z lematu \ref{lemat:1} $G$ ma cykl, a cykl $\subset$ obwód.
Weźmy najdłuższy obwód w $G$ i nazwijmy $C$. $E(C) \neq E(G)$ bo w przeciwnym przypadku $C$ byłby obwodem Eulera w $G$.
\\Rozważmy $G-E(C)$ zawiera jakieś krawędzie. Zastanówmy się nad parzystością stopni $G-E(C)$. W $G-E(C)$ każdy wierzchołek jest parzystego stopnia. czyli każda składowa o nieparzystym zbiorze krawędzi\footnote{$G-E(C)$ nie musi być grafem spójnym} $G-E(C)$ nie jest kontrprzykładem ($G$ był najmniejszy) zatem ma obwód Eulera $C'$.\\
Istnieje wierzchołek $x\in V(C) \cap V(C')$ bo w przeciwnym przypadku $G$ nie byłby spójny.\\
Obwód $xCxC'x$ jest dłuższy niż $C$, a $G$ maił być najdłuższy (o większej liczbie krawędzi). Sprzeczność z definicją $C$.
\end{proof}
%TODO Sprawdzić ten dowód
\end{tw}

\subsection{Algorytm znajdowania obwodu Eulera w grafie}
\begin{lemat}
Jeżeli graf $G$ jest spójny i ma wszystkie wierzchołki parzystego stopnia, to nie ma mostu
\begin{proof}
Przypuśćmy przeciwnie.\\
Niech $e = xy$ będzie mostem w $G$. $G-e$ ma dokładnie dwie składowe $G_x, G_y$. W $G_x$ wszystkie wierzchołki oprócz $X$ są parzystego stopnia. Sprzeczność z zasadą, że suma stopni w grafie jest parzysta (suma stopni jest dwa razy większa od liczby krawędzi)
\end{proof}
\end{lemat}

\subsubsection{Algorytm Fleury'ego}
\begin{enumerate}
\item Wybieram wierzchołek startowy $v_0, ~ w = v_0$
\item Załóżmy, że skonstruowaliśmy już ścieżkę $w=v_0e_1v_1 \dots e_iv_i$ wybieram krawędź $e_{i+1}$ taką że
	\begin{enumerate}
		\item $e_{i+1} \in E(G) \smallsetminus \{e_1, \dots, e_i\}$
		\item $e_{i+1}$ jest incydentna z $v_i$
		\item o ile to możliwe $e_{i+1}$ nie jest mostem w $G_i = G - \{e_1, \dots, e_i\}$\label{point:1}
	\end{enumerate}
\item STOP kiedy krok 2 nie może się wykonać
%TODO Dodać przykład
	\end{enumerate}
\end{enumerate}

\begin{tw}
Jeżeli $G$ ma obwód Eulera to algorytm Fleury'ego znajdzie taki obwód
\begin{proof}
$w_n = v_0e_1v_1 \dots e_nv_n$ ścieżka skonstruowana przez algorytm Fleury'ego. $w_n$ nie zawiera żadnej krawędzi więcej niż raz. W momencie zatrzymania się algorytmu stopnień $v_n$ w grafie $G-\{e_1, \dots, e_n\}$ jest równy $0$.\\
Jeżeli $v_n = v_0$ to ma nieparzysty stopień, a ma mieć $0$. Czyli $v_n = v_0$\footnote{gdyby tak nie było oznaczałoby to  że wyszliśmy o jeden raz więcej niż weszliśmy} Czyli $\deg_{G-\{e_1, \dots, e_n\}}v_n$ jest nieparzysty. Sprzeczność bo zero jest parzyste}.\\
Wystarczy pokazać, że $E(w_n)=E(G)$.\\
Przypuśćmy, że nie jest. $G_n = G-E(w_n)$ ma niepusty zbiór krawędzi.
\\ Niech $S$ - zbiór wierzchołków dodatniego stopnia w $G_n$.
\\ Niech $\overline{S}$ - zbiór wszystkich wierzchołków zerowego stopnia. $\overline{S}=V(G)-S$
\\ Niech $m$ będzie największym indeksem takim że $V_m \in S$ a $V_{m+1} = e_{m+1}$
%TODO Dodać obrazek
\\ Czyli krawędź $e_{m+1}$ jest mostem w $G_m = G - \{e_1, \dots, e_m\}$ bo jest jedyną krawędzią między $S$ i $\overline{S}$. Z punkty \ref{point:1} wynika że wszystkie krawędzie w $G_m$ o końcu $v_m$ są mostami.
\\ Rozważmy $G_m[S]$\footnote{graf indukowany}, w którym wszystkie wierzchołki są parzystego stopnia a $v_m$ jest końcem mostu. Sprzeczność z lematem.
%TODO Porównać z http://pl.wikipedia.org/wiki/Algorytm_Fleury%27ego#Dow.C3.B3d_poprawno.C5.9Bci_algorytmu
\end{proof}

\end{tw}


\end{document}
